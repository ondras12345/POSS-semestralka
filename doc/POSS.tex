\documentclass[a4paper,11pt]{article}
\usepackage{cmap}  % jinak [T1]{fontenc} rozbije hledani a kopirovani z PDF
\usepackage[T1]{fontenc}
\usepackage[utf8]{inputenc}
\usepackage[czech]{babel}

\usepackage[
    a4paper,
    %showframe,
    includefoot,bottom=20mm
]{geometry}
\usepackage{titling}
\usepackage{booktabs}
\usepackage{siunitx}
%\usepackage{array}
\usepackage{fancyhdr}
%\usepackage[shortlabels]{enumitem}
%\usepackage{amsmath}
%\usepackage{mathtools}
%\usepackage{gnuplottex}
%\usepackage{pdfpages}
%\usepackage{placeins}  % \FloatBarrier
\usepackage{float}
%\usepackage{caption}
%\usepackage{subcaption}
\usepackage{listings}
%\usepackage{bm}
\usepackage[hidelinks]{hyperref}
%\usepackage[czech]{varioref}
%\usepackage[backend=biber,style=iso-numeric]{biblatex}
%\addbibresource{reference.bib}

\usepackage{csquotes}

\usepackage{graphicx}
\graphicspath{{figures/}}

\usepackage{dirtree}

\usepackage[disable]{todonotes}  % [disable]

%\usepackage[EFvoltages,siunitx,european, american inductors]{circuitikz}
\usepackage{pict2e}
\usepackage{tikz}
\usetikzlibrary{automata, positioning, arrows}
%\usetikzlibrary{shapes,positioning,calc,fit,arrows.meta}
%\usetikzlibrary{fit}

% https://www3.nd.edu/~dchiang/teaching/theory/2018/www/tikz_tutorial.pdf
\tikzset{
    ->, % makes the edges directed
    >=stealth', % makes the arrow heads bold
    node distance=2.5cm and 1.0cm,  % vertical horizontal
    every state/.style={thick, fill=gray!10, rectangle, rounded corners},
    every loop/.style={looseness=2},
}

\usepackage[toc,page]{appendix}
\renewcommand{\appendixname}{Příloha}
\renewcommand{\appendixtocname}{Přílohy}
\renewcommand{\appendixpagename}{Přílohy}

% Roztrhana matika
%\allowdisplaybreaks


\sisetup{
    output-decimal-marker = {,},
    list-separator = {; },
    list-final-separator = { a~},
    list-pair-separator = { a~},
    group-digits,
    group-minimum-digits=4,
    range-phrase=\text{--},
    range-units=single,
    exponent-product=\ensuremath{\cdot},
    uncertainty-mode=separate,
    output-complex-root=\iu,
}


\lstset{language=C++,
    basicstyle={\fontfamily{lmtt}\selectfont},
    columns=fullflexible,
    showlines=true,
    breaklines=true,
    postbreak=\mbox{$\hookrightarrow$\space},
    showstringspaces=false,
    xleftmargin=1em,
    literate=%
      {---}{{\llap{\mdseries-{-}-}}}3
      {>}{{\textgreater}}1
      {>-}{{\textgreater-}}2
      {|}{{\textbar}}1
      {|-}{{\textbar-}}2
      % make dashes shorter
      {-}{{-}}1
      {-\ }{{\mdseries-\ }}2
      {°}{\textdegree}1
      {á}{{\'a}}1
      {í}{{\'i}}1
      {é}{{\'e}}1
      {ý}{{\'y}}1
      {ú}{{\'u}}1
      {ó}{{\'o}}1
      {ě}{{\v{e}}}1
      {š}{{\v{s}}}1
      {č}{{\v{c}}}1
      {ř}{{\v{r}}}1
      {ž}{{\v{z}}}1
      {ď}{{\v{d}}}1
      {ť}{{\v{t}}}1
      {ň}{{\v{n}}}1
      {ů}{{\r{u}}}1
      {Á}{{\'A}}1
      {Í}{{\'I}}1
      {É}{{\'E}}1
      {Ý}{{\'Y}}1
      {Ú}{{\'U}}1
      {Ó}{{\'O}}1
      {Ě}{{\v{E}}}1
      {Š}{{\v{S}}}1
      {Č}{{\v{C}}}1
      {Ř}{{\v{R}}}1
      {Ž}{{\v{Z}}}1
      {Ď}{{\v{D}}}1
      {Ť}{{\v{T}}}1
      {Ň}{{\v{N}}}1
      {Ů}{{\r{U}}}1,
}


\lstdefinestyle{terminal}{
    language=,
    basicstyle={\fontfamily{lmtt}\selectfont\footnotesize},
    frame=single,
    frameround=tttt,
    columns=flexible,
}

\newcommand\conf[1]{\texttt{conf.\detokenize{#1}}}
\newcommand\CLIcommand[1]{\texttt{\detokenize{#1}}}

% headers and footers
\setlength{\headheight}{26pt} % prevents a warning most likely caused by diacritics; two author names
\pagestyle{fancy}
%\renewcommand{\headrulewidth}{0pt}
\fancyhf{} % clears header and footer, otherwise "plain" will appear
%\fancyhead{} % only clears the header
%\lhead{\theauthor}
\lhead{\theauthor}
\chead{\thetitle}
\rhead{\thepage}
\fancyhfoffset{0pt} % fixes width after \newgeometry


\author{Ondřej Sluka}
\date{2024-05-17}
\title{Semestrální projekt KKY/POSS}


\begin{document}
\maketitle
\tableofcontents

\section{Zadání}
Cílem této práce je vytvořit firmware pro mikrokontrolér ATmega 2560 na
robotické platformě mBot Ranger, který umožní autonomní zmapování bludiště
a~následný rychlý průjezd optimální trasou.

\section{Řešení}
Omezíme se na řešení problému mapování bludiště \emph{bez cyklů}.

\subsection{Struktura repositáře se zdrojovým kódem}
Firmware je vyvinut v~jazyce C/C++ s~využitím frameworku Arduino, pro sestavení
používáme nástroj \texttt{platformio}.
Adresář se zdrojovým kódem je rozdělen do několika podadresářů:
\begin{figure}[H]
    % musi to byt float, neumi se rozdelit na vice stranek
    \dirtree{%
        .1 .
        .2 doc/\DTcomment{adresář se zdrojovými kódy tohoto dokumentu}.
        .2 lib/\DTcomment{adresář s~knihovnami specifickými pro tento projekt}.
        .3 MeRGBLineFollower180motorAuriga/\DTcomment{ovladač pro RGB modul sledování čáry, staženo z~courseware}.
        .3 PID/\DTcomment{naše vlastní knihovna implementující PID regulátory, oddělena pro snazší testování}.
        .3 perf\_counter/\DTcomment{vlastní knihovna pro měření doby provádění částí programu}.
        .2 src/\DTcomment{adresář se zdrojovými kódy firmware}.
        .3 cli.\{h,cpp\}\DTcomment{příkazový řádek na sériovém portu}.
        .3 conf.\{h,cpp\}\DTcomment{ladění parametrů bez nutnosti resetu}.
        .3 crossroad.\{h,cpp\}\DTcomment{podpůrné funkce pro detekci křižovatek}.
        .3 debug.\{h,cpp\}\DTcomment{podpora pro volitelné ladicí výpisy}.
        .3 encoder.\{h,cpp\}\DTcomment{snímání polohy z~enkodérů}.
        .3 imu.\{h,cpp\}\DTcomment{obsluha gyroskopu}.
        .3 line\_follower.\{h,cpp\}\DTcomment{regulátor pro sledování čáry; detekce křižovatek}.
        .3 main.cpp\DTcomment{hlavní smyčka, stavový automat, \ldots}.
        .3 maze.\{h,cpp\}\DTcomment{mapování bludiště a~finální průjezd}.
        .3 motor.\{h,cpp\}\DTcomment{ovládání motorů}.
        .3 robot.\{h,cpp\}\DTcomment{definice stavů pro stavový automat}.
        .3 turn.\{h,cpp\}\DTcomment{zatáčení}.
        .2 test/\DTcomment{adresář se zdrojovými kódy automatických testů}.
        .3 test\_PID/\DTcomment{testy knihovny PID}.
        .4 test\_PID.m\DTcomment{referenční implementace PID regulátoru -- výukový materiál předmětu KKY/LS2}.
        .4 \ldots.
        .3 test\_encoder/\DTcomment{testy výpočtů vzdálenosti v~kompilační jednotce \texttt{encoder}}.
        .3 test\_maze/\DTcomment{testy algoritmů týkajících se průchodu bludištěm}.
        .4 test\_maze.gpi\DTcomment{gnuplot skript pro vygenerování gif animací průchodů bludištěm}.
        .4 maze.xcf\DTcomment{mapa bludiště -- bitmapový obrázek ve formátu grafického editoru GIMP}.
        .4 maze.data\DTcomment{mapa bludiště -- binární formát}.
        .4 \ldots.
        .2 Makefile\DTcomment{konfigurace build systému \texttt{make}}.
        .2 platformio.ini\DTcomment{konfigurace \texttt{platformio}}.
    }
\end{figure}

Kompilace se spouští příkazem
\begin{lstlisting}[style=terminal]
$ make
pio run
Processing megaatmega2560 (platform: atmelavr; board: megaatmega2560; framework: arduino)
-------------------------------------------------------------------------------
[...]
RAM:   [====      ]  44.4% (used 3637 bytes from 8192 bytes)
Flash: [==        ]  18.9% (used 48010 bytes from 253952 bytes)
[...]
\end{lstlisting}
Příkaz \texttt{make check} provádí statickou analýzu kódu, \texttt{make test}
spouští automatické testy (testy běží nativně na počítači, není potřeba mít
přístup k mikrokontroléru architektury AVR), \texttt{make upload} nahraje
program do mikrokontroléru pomocí bootloaderu.


\subsection{CLI}
Firmware robota nabízí rozhraní s~příkazovým řádkem, kterým je možné
ovládat a~samostatně testovat všechny jeho funkce, vyčítat data ze
senzorů a~ladit parametry. Příklad použití je otištěn
v~příloze~\ref{app:CLI}.


\subsection{Stavový automat}
Hlavní stavový automat v~kompilační jednotce \texttt{main} řídí vysokoúrovňové
funkce robota. Automat přijímá povely od uživatele (stisky nárazníků, nastavení
stavu příkazem \texttt{state} v~CLI). Stavový diagram je na
obrázku~\ref{fig:stavovy diagram}. Některé stavy (např. \texttt{line\_follow})
je možné dosáhnout pouze z~CLI. Některé stavy se používají spíše jako příkazy
(stavový automat v~nich nesetrvá).

\begin{figure}[htbp]
    \centering
    \begin{tikzpicture}
        \node[state, initial, accepting] (s_emergency) {\texttt{emergency}};
        \node[state, below right=of s_emergency] (s_maze_follow) {\texttt{maze\_follow}};
        \node[state, below=of s_maze_follow] (s_maze_following) {\texttt{maze\_following}};
        \node[state, below left=of s_emergency] (s_map_start) {\texttt{map\_start}};
        \node[state, below=of s_map_start] (s_map_mapping) {\texttt{map\_mapping}};
        \node[state, below left=of s_maze_following] (s_idle) {\texttt{idle}};
        \node[state, left=of s_idle] (s_stop) {\texttt{stop}};
        \node[state, right=of s_idle] (s_finish) {\texttt{finish}};
        \node[state, left=of s_map_start] (s_line_follow) {\texttt{line\_follow}};
        \node[state, below=of s_line_follow] (s_line_following) {\texttt{line\_following}};
        \draw
            (s_emergency)       edge[above, sloped] node{pravý nárazník}            (s_maze_follow)
            (s_emergency)       edge[above, sloped] node{levý nárazník}             (s_map_start)
            (s_maze_follow)     edge[above, sloped] node[text width=2cm]{na čáře a~má mapu} (s_maze_following)
            (s_maze_follow)     edge[bend right, above, sloped] node{jinak}         (s_idle)
            (s_map_start)       edge                                                (s_map_mapping)
            (s_maze_following)  edge[loop right]                                    (s_maze_following)
            (s_map_mapping)     edge[loop left]                                     (s_map_mapping)
            (s_stop)            edge                                                (s_idle)
            (s_map_mapping)     edge[above, sloped, bend right] node{chyba}         (s_emergency)
            (s_maze_following)  edge[above, sloped, bend left] node{chyba}          (s_emergency)
            (s_map_mapping)     edge[above, sloped, bend right] node{v cíli}        (s_finish)
            (s_maze_following)  edge[above, sloped] node{v cíli}                    (s_finish)
            (s_finish)          edge[below, sloped, bend left, out=270, in=270, looseness=0.8] node{pravý nárazník} (s_maze_follow)
            (s_line_follow)     edge                                                (s_line_following)
            (s_line_following)  edge[loop left]                                     (s_line_following)
            (s_line_following)  edge[above, sloped, bend right] node{křižovatka}    (s_stop)
            (s_idle)            edge[above, sloped] node{pravý nárazník}            (s_maze_follow)
            (s_idle)            edge[above, sloped] node{levý nárazník}             (s_map_start)
            (s_idle)            edge[loop below]                                    (s_idle)
            (s_finish)          edge[loop below]                                    (s_finish)
            (s_emergency)       edge[loop right]                                    (s_emergency)
            ;
    \end{tikzpicture}
    \caption{Stavový diagram hlavního stavového automatu}
    \label{fig:stavovy diagram}
\end{figure}


\subsection{Detekce křižovatek}
Detekce křižovatek využívá dat ze čtyř senzorů čáry a~informaci o~ujeté
vzdálenosti z~enkodérů na motorech. V~tabulce~\ref{tab:krizovatky} jsou
otištěny uvažované typy křižovatek a~odpovídající hodnoty výčtového datového
typu \verb!crossroad_t! z~\texttt{crossroad.h}. Detekci křižovatek provádí
\texttt{line\_follower.cpp}, který nabízí několik funkcí pro zjištění aktuální
křižovatky. Úplná detekce je provedena až po přejetí křižovatky
o~\conf{cr_delay_mm} milimetrů.

\begin{table}[htbp]
    \centering
    \caption{Uvažované typy křižovatek}
    \label{tab:krizovatky}
    \setlength{\unitlength}{3mm}
    \linethickness{.3mm}
    \begin{tabular}{c>{\ttfamily\catcode`_=12}c>{\ttfamily}cl}
        \toprule
        křižovatka & \verb!crossroad_t! & znak & popis \\
        \midrule
                   & cr_0 & '0' & nevím / neinicializováno / mimo čáru / nevalidní \\
        \begin{picture}(1,1)
            \Line(0.5,0)(0.5,1)
        \end{picture}
                   & cr_I & 'I' & \\
        \begin{picture}(1,1)
            \Line(.5,0)(.5,.5)
            \Line(.5,.5)(1,.5)
        \end{picture}
                   & cr_G & 'G' & \\
        \begin{picture}(1,1)
            \Line(.5,0)(.5,.5)
            \Line(0,.5)(.5,.5)
        \end{picture}
                   & cr_7 & '7' & \\
        \begin{picture}(1,1)
            \Line(.5,0)(.5,.5)
            \Line(0,.5)(1,.5)
        \end{picture}
                   & cr_T & 'T' & \\
        \begin{picture}(1,1)
            \Line(.5,0)(.5,1)
            \Line(.5,.5)(1,.5)
        \end{picture}
                   & cr_E & 'E' & \\
        \begin{picture}(1,1)
            \Line(.5,0)(.5,1)
            \Line(0,.5)(.5,.5)
        \end{picture}
                   & cr_3 & '3' & \\
        \begin{picture}(1,1)
            \Line(.5,0)(.5,1)
            \Line(0,.5)(1,.5)
        \end{picture}
                   & cr_X & 'X' & \\
        \begin{picture}(1,1)
            \Line(.5,0)(.5,.5)
        \end{picture}
                   & cr_i & 'i' & slepý konec \\
        \begin{picture}(1,1)
            \polygon*(0,0)(0,1)(1,1)(1,0)
        \end{picture}
                   & cr_F & 'F' & cíl \\
        \bottomrule
    \end{tabular}
\end{table}


\subsection{Sledování čáry}
Pro sledování čáry používáme P regulátor se zesílením \conf{line_Kp} a~akčním
zásahem saturovaným na hodnotě \conf{line_umax}, přičemž regulační odchylku
čteme přímo z~modulu čidla čáry, akční zásah přičítáme k~požadované rychlosti
levého pásu a~odčítáme od požadované rychlosti pravého pásu.


\subsection{Otáčení}
Otáčení robota využívá zpětné vazby z~gyroskopu a~můžeme je spustit z~CLI dvěma
způsoby: \CLIcommand{turn <target>} a~\CLIcommand{turn line <target>}.

\CLIcommand{turn} otočí robota o~\texttt{<target>} stupňů, přičemž kladné
hodnoty otáčí doprava.

\CLIcommand{turn line} funguje podobně, ale v~okolí \conf{turn_line_tolerance}
stupňů kolem požadované hodnoty začne robot hledat čáru (minimum regulační
odchylky pro sledování čáry). Když je minimum nalezeno, je aktuální úhel
z~gyroskopu nastaven jako nová cílová hodnota.

Otáčení je řízeno PID regulátorem s~parametry \conf{turn_Kp}, \conf{turn_Ki},
\ldots. Bylo nutné implementovat speciální verzi regulátoru (viz
knihovna \texttt{PID}), neboť gyroskop vrací úhel natočení kolem osy
Z~v~rozsahu \numrange[range-phrase={ \text{až} }]{-180}{180}.
Pokud je regulační odchylka i~akční zásah v~absolutní hodnotě menší než
nastavené prahové hodnoty, je otáčení ukončeno.

Autonomní funkce robota používají pouze ekvivalent příkazu \CLIcommand{turn
line}, při otáčení vždy hledají čáru.


\subsection{Mapování}
Mapování probíhá metodou backtracking. Pracujeme se zásobníkem reprezentujícím
cestu od startu k~aktuální pozici robota tvořenou záznamy o~typu křižovatky,
směru průjezdu a~vzdálenosti od předchozí křižovatky.
Pomocí CLI můžeme s~mapou manipulovat pomocí následujících příkazů:
\begin{lstlisting}[style=terminal,tabsize=10,columns=fixed]
sluka:$ help
[...]
maze_pop: Pop a node off the route
maze_print: Print current maze route
maze_push: Push a node onto the route
[...]

sluka:$ maze_print
maze_route_current (bottom to top)
          crossroad	direction	distance [mm]
maze_push X	I	197
maze_push 3	I	302
maze_push X	L	310
maze_push G	R	333
maze_push 7	L	280
maze_push G	R	658
maze_push E	R	1342
maze_push X	I	1327
maze_push 3	L	307
maze_push 3	L	645
maze_push X	R	1330
maze_push F	I	319

sluka:$
\end{lstlisting}
(Výše vypsaná cesta je finální cesta do cíle získaná procesem mapování.)

Soubor \texttt{test/test\_maze/maze\_map.gif}\footnote{Nepodařilo se nám vložit
animovaný GIF do PDF souboru tak, aby jej úspěšně zobrazily všechny
prohlížeče.} obsahuje animaci procesu mapování vygenerovanou testem
\texttt{test\_maze}.


\subsection{Finální průjezd}
Při finálním průjezdu se obsah zásobníku čte zdola nahoru a~záznam se porovnává
s~naposledy detekovanou křižovatkou. Pokud se informace neshodují, robot
zastaví a~přejde do nouzového režimu. V~opačném případě robot na křižovatce
odbočí podle instrukce v~zásobníku.

V~okolí křižovatky jede robot rychlostí \conf{base_speed}. Pokud je od
nejbližší křižovatky vzdálen nejméně \conf{fast_offset_mm}, nebo pokud na
nejbližší křižovatce jel (či pojede) rovně, jede rychlostí \conf{fast_speed}.


\section{Závěr}
Podařilo se vyvinou firmware, se kterým robot úspěšně vyřešil úlohu. Finální
průjezd trval \qty{22}{\second}. Firmware obsahuje bohatou sadu nástrojů pro
diagnostiku a~ladění parametrů bez nutnosti resetu; funkce jeho
nejdůležitějších částí je kontrolována automatickými testy.


\todo[inline]{pozor na datum na titulni strance}
\listoftodos

%\printbibliography

\clearpage
\begin{appendices}
    \section{Příklad použití příkazového řádku}
    \label{app:CLI}
    \begin{lstlisting}[style=terminal]
sluka:$ help

---- Shortcut Keys ----

Ctrl-A : Jumps the cursor to the beginning of the line.
Ctrl-E : Jumps the cursor to the end of the line.
Ctrl-D : Log Out.
Ctrl-R : Reverse-i-search.
Pg-Up  : History search backwards and auto completion.
Pg-Down: History search forward and auto completion.
Home   : Jumps the cursor to the beginning of the line.
End    : Jumps the cursor to the end of the line.

---- Available commands ----

conf: Get/set config
debug: Enable/disable debug
encoder: Read rotary encoders
imu: Get IMU state
line: Read line follower
maze_pop: Pop a node off the route
maze_print: Print current maze route
maze_push: Push a node onto the route
motor_move: Set motor output (nonlinear)
motor_move_lin: Set motor output (linearized)
perf: Print perf counters
state: Get/set state machine state
turn: Get / set turn state
uptime: Returns the time passed since the program started.

sluka:$ conf
Usage: conf [name value]

configuration:
conf mm_per_pulse 0.20
conf base_speed 60
conf fast_speed 220
conf map_speed 60
conf fast_offset_mm 100
conf line_Kp 0.60
conf line_umax 60
conf turn_Kp 15.00
conf turn_Ki 30.00
conf turn_Kd 0.00
conf turn_Tf 0.50
conf turn_Tt 15.00
conf turn_target 90
conf turn_line_tolerance 55
conf line_debounce 2
conf dead_end_dist 30
conf min_cr_dist 80
conf cr_delay_mm 10

sluka:$ debug
crossroad: 0
encoder: 0
maze_follow: 0
map: 0

sluka:$ debug encoder 1
crossroad: 0
encoder: 1
maze_follow: 0
map: 0

sluka:$ # ručně pohnu s pásy
sluka:$ [D] went 10mm
[D] went 10mm
[D] went 10mm
[D] went 10mm

sluka:$ imu
imu:
  angle_X: -0.37
  angle_Y: 0.01
  angle_Z: -1.25

sluka:$ turn 90
turning 90
turning: 1
target: 89.07
angle_Z: -0.93
usage: turn [[line] target]

sluka:$ imu
imu:
  angle_X: -0.03
  angle_Y: -0.31
  angle_Z: 91.72

sluka:$ turn -90
turning -90
turning: 1
target: 1.88
angle_Z: 91.88
usage: turn [[line] target]

sluka:$ imu
imu:
  angle_X: -0.20
  angle_Y: 0.00
  angle_Z: -0.22

sluka:$ perf
cli     2800
line_follower   1368
state_machine   44
imu     2840
turn    36
maze    44

sluka:$ state
state: idle
emergency: 0

sluka:$ maze_print
maze_route_current (bottom to top)
        crossroad       direction       distance [mm]

sluka:$ maze_push X I 100

sluka:$ maze_push 3 I 300

sluka:$ maze_push X L 300

sluka:$ maze_print
maze_route_current (bottom to top)
        crossroad       direction       distance [mm]
maze_push X     I       100
maze_push 3     I       300
maze_push X     L       300

sluka:$
    \end{lstlisting}
\end{appendices}

\end{document}
